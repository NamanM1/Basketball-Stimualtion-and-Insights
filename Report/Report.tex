% Options for packages loaded elsewhere
\PassOptionsToPackage{unicode}{hyperref}
\PassOptionsToPackage{hyphens}{url}
%
\documentclass[
]{article}
\usepackage{amsmath,amssymb}
\usepackage{lmodern}
\usepackage{iftex}
\ifPDFTeX
  \usepackage[T1]{fontenc}
  \usepackage[utf8]{inputenc}
  \usepackage{textcomp} % provide euro and other symbols
\else % if luatex or xetex
  \usepackage{unicode-math}
  \defaultfontfeatures{Scale=MatchLowercase}
  \defaultfontfeatures[\rmfamily]{Ligatures=TeX,Scale=1}
\fi
% Use upquote if available, for straight quotes in verbatim environments
\IfFileExists{upquote.sty}{\usepackage{upquote}}{}
\IfFileExists{microtype.sty}{% use microtype if available
  \usepackage[]{microtype}
  \UseMicrotypeSet[protrusion]{basicmath} % disable protrusion for tt fonts
}{}
\makeatletter
\@ifundefined{KOMAClassName}{% if non-KOMA class
  \IfFileExists{parskip.sty}{%
    \usepackage{parskip}
  }{% else
    \setlength{\parindent}{0pt}
    \setlength{\parskip}{6pt plus 2pt minus 1pt}}
}{% if KOMA class
  \KOMAoptions{parskip=half}}
\makeatother
\usepackage{xcolor}
\IfFileExists{xurl.sty}{\usepackage{xurl}}{} % add URL line breaks if available
\IfFileExists{bookmark.sty}{\usepackage{bookmark}}{\usepackage{hyperref}}
\hypersetup{
  pdftitle={Basketball Simulation and Insights},
  hidelinks,
  pdfcreator={LaTeX via pandoc}}
\urlstyle{same} % disable monospaced font for URLs
\usepackage[margin=1in]{geometry}
\usepackage{graphicx}
\makeatletter
\def\maxwidth{\ifdim\Gin@nat@width>\linewidth\linewidth\else\Gin@nat@width\fi}
\def\maxheight{\ifdim\Gin@nat@height>\textheight\textheight\else\Gin@nat@height\fi}
\makeatother
% Scale images if necessary, so that they will not overflow the page
% margins by default, and it is still possible to overwrite the defaults
% using explicit options in \includegraphics[width, height, ...]{}
\setkeys{Gin}{width=\maxwidth,height=\maxheight,keepaspectratio}
% Set default figure placement to htbp
\makeatletter
\def\fps@figure{htbp}
\makeatother
\setlength{\emergencystretch}{3em} % prevent overfull lines
\providecommand{\tightlist}{%
  \setlength{\itemsep}{0pt}\setlength{\parskip}{0pt}}
\setcounter{secnumdepth}{-\maxdimen} % remove section numbering
\ifLuaTeX
  \usepackage{selnolig}  % disable illegal ligatures
\fi

\title{Basketball Simulation and Insights}
\author{}
\date{\vspace{-2.5em}}

\begin{document}
\maketitle

\{=html\}

Members

Nandini Bhattad 220693

Ankita Goyal 231080016

Gunavant Thakare 231080040

Naman Manchanda 231080061

Dr.~Dootika Vats

Data Science Lab 1

\hypertarget{acknowledgments}{%
\section{Acknowledgments}\label{acknowledgments}}

I would like to express my sincere gratitude to Dr.~Dootika Vats for her
invaluable guidance and support throughout the Data Science Lab 1
project. Her expertise and mentorship have played a pivotal role in
shaping the direction of this project.

Dr.~Vats' insightful feedback, encouragement, and dedication to
fostering a collaborative learning environment have been instrumental in
enhancing my understanding of data science concepts and methodologies.

I am truly thankful for the opportunity to work under Dr.~Vats'
supervision, and I appreciate her commitment to excellence in teaching
and research. This project has been a rewarding experience, largely due
to her mentorship.

\begin{center}\rule{0.5\linewidth}{0.5pt}\end{center}

\hypertarget{table-of-contents}{%
\section{Table of Contents}\label{table-of-contents}}

\begin{itemize}
\tightlist
\item
  \protect\hyperlink{introduction}{1. Introduction}
\item
  \protect\hyperlink{methodology}{2. Methodology}
\item
  \protect\hyperlink{data}{3. Data}
\item
  \protect\hyperlink{analysis}{4. Analysis}
\item
  \protect\hyperlink{results}{3. Results and Conclusion}
\item
  \protect\hyperlink{references}{References}
\end{itemize}

\begin{center}\rule{0.5\linewidth}{0.5pt}\end{center}

\hypertarget{introduction}{%
\section{Introduction}\label{introduction}}

The Basketball Simulation and Insights project delve into the dynamic
world of basketball analytics, aiming to leverage data science
methodologies to gain valuable insights into player performance, team
strategies, and game outcomes. Basketball, as a highly dynamic and
fast-paced sport, generates a wealth of data during each match, from
player statistics to in-game events, providing a rich landscape for
analysis.

The primary objective of this project is to develop a comprehensive
basketball simulation model that can replicate real-game scenarios,
allowing us to study various factors influencing team and player
performance. By harnessing the power of data science and statistical
modeling, we seek to unravel patterns, trends, and key indicators that
contribute to success on the basketball court.

In addition to the simulation aspect, this project aims to provide
actionable insights for coaches, analysts, and enthusiasts. Through
exploratory data analysis and machine learning techniques, we aspire to
identify critical performance metrics, strategic plays, and potential
areas for improvement. The insights derived from this project have the
potential to inform coaching decisions, optimize player training
regimens, and contribute to a deeper understanding of the sport.

The project will encompass a diverse range of data sources, including
player statistics, game play-by-plays, and team dynamics. Our analytical
approach will involve data cleaning, feature engineering, and the
implementation of machine learning algorithms to model and simulate
basketball scenarios.

As we embark on this basketball analytics journey, we anticipate
uncovering new perspectives on the sport, fostering innovation in
coaching strategies, and contributing to the growing field of sports
analytics. The combination of simulation and insights aims to not only
enhance our understanding of basketball dynamics but also provide
practical applications for teams and enthusiasts passionate about the
game.

\begin{center}\rule{0.5\linewidth}{0.5pt}\end{center}

\hypertarget{methodology}{%
\section{Methodology}\label{methodology}}

\hypertarget{data-collection}{%
\subsubsection{Data Collection}\label{data-collection}}

The foundation of this project involves the extraction of comprehensive
player statistics from the NBA 2022-23 season. We employed web scraping
techniques to gather data from the Basketball Reference website
(\url{https://www.basketball-reference.com}). The website provides
detailed and up-to-date information on NBA player statistics, team
performance, and game results.

\hypertarget{data-source}{%
\subsubsection{Data Source}\label{data-source}}

The primary data source for this project was the individual player
statistics pages on Basketball Reference. We focused on collecting
information for all 679 players who participated in the NBA during the
2022-23 season.

\hypertarget{web-scraping}{%
\subsubsection{Web Scraping}\label{web-scraping}}

We utilized the \texttt{rvest} package in R for web scraping, allowing
us to programmatically extract player statistics. The scraping process
involved navigating through player pages, retrieving summary statistics,
and storing the data for further analysis.

\begin{center}\rule{0.5\linewidth}{0.5pt}\end{center}

\hypertarget{data}{%
\section{Data}\label{data}}

The following variables were used, Rk (Rank): The player's rank or
position in the specified context, often indicating their rank among all
players in a certain category or statistical measure.

Player: The name of the basketball player.

Pos (Position): The player's position on the basketball court, such as
guard (G), forward (F), or center (C).

\begin{enumerate}
\def\labelenumi{\arabic{enumi}.}
\item
  Age: The age of the player during the specified season.
\item
  Tm (Team): The abbreviation or code representing the team the player
  is associated with during the specified season.
\item
  G (Games Played): The number of games the player participated in
  during the season.
\item
  GS (Games Started): The number of games in which the player was in the
  starting lineup.
\item
  MP (Minutes Per Game): The average number of minutes the player spent
  on the court per game.
\item
  FG (Field Goals Made): The total number of successful field goals made
  by the player.
\item
  FGA (Field Goals Attempted): The total number of field goal attempts
  by the player.
\item
  FG\% (Field Goal Percentage): The shooting accuracy of the player,
  calculated as (FG / FGA) * 100.
\item
  3P (Three-Pointers Made): The total number of successful three-point
  shots made by the player.
\item
  3PA (Three-Pointers Attempted): The total number of three-point shot
  attempts by the player.
\item
  3P\% (Three-Point Percentage): The three-point shooting accuracy of
  the player, calculated as (3P / 3PA) * 100.
\item
  2P (Two-Pointers Made): The total number of successful two-point field
  goals made by the player.
\item
  2PA (Two-Pointers Attempted): The total number of two-point field goal
  attempts by the player.
\item
  2P\% (Two-Point Percentage): The two-point shooting accuracy of the
  player, calculated as (2P / 2PA) * 100.
\item
  eFG\% (Effective Field Goal Percentage): A modified field goal
  percentage that accounts for the added value of three-pointers,
  calculated as (FG + 0.5 * 3P) / FGA * 100.
\item
  FT (Free Throws Made): The total number of successful free throws made
  by the player.
\item
  FTA (Free Throws Attempted): The total number of free throw attempts
  by the player.
\item
  FT\% (Free Throw Percentage): The free throw shooting accuracy of the
  player, calculated as (FT / FTA) * 100.
\item
  ORB (Offensive Rebounds): The total number of offensive rebounds
  grabbed by the player.
\item
  DRB (Defensive Rebounds): The total number of defensive rebounds
  grabbed by the player.
\item
  TRB (Total Rebounds): The total number of rebounds (both offensive and
  defensive) grabbed by the player.
\item
  AST (Assists): The total number of assists made by the player.
\item
  STL (Steals): The total number of steals made by the player.
\item
  BLK (Blocks): The total number of shots blocked by the player.
\item
  TOV (Turnovers): The total number of turnovers committed by the
  player.
\item
  PF (Personal Fouls): The total number of personal fouls committed by
  the player.
\item
  PTS (Points): The total number of points scored by the player.
\end{enumerate}

\begin{center}\rule{0.5\linewidth}{0.5pt}\end{center}

\hypertarget{analysis}{%
\section{Analysis}\label{analysis}}

\begin{center}\rule{0.5\linewidth}{0.5pt}\end{center}

\hypertarget{results}{%
\section{Results}\label{results}}

\begin{center}\rule{0.5\linewidth}{0.5pt}\end{center}

\hypertarget{references}{%
\section{References}\label{references}}

\begin{enumerate}
\def\labelenumi{\arabic{enumi}.}
\item
  \emph{Basketball Reference.} (n.d.). Retrieved from
  \url{https://www.basketball-reference.com}
\item
  \emph{NBA Official Website.} (n.d.). Retrieved from
  \url{https://www.nba.com}
\end{enumerate}

\end{document}
